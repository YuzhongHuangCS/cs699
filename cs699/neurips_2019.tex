\documentclass{article}

% if you need to pass options to natbib, use, e.g.:
%     \PassOptionsToPackage{numbers, compress}{natbib}
% before loading neurips_2019

% ready for submission
% \usepackage{neurips_2019}

% to compile a preprint version, e.g., for submission to arXiv, add add the
% [preprint] option:
%     \usepackage[preprint]{neurips_2019}

% to compile a camera-ready version, add the [final] option, e.g.:
     \usepackage[final]{neurips_2019}

% to avoid loading the natbib package, add option nonatbib:
%     \usepackage[nonatbib]{neurips_2019}

\usepackage[utf8]{inputenc} % allow utf-8 input
\usepackage[T1]{fontenc}    % use 8-bit T1 fonts
\usepackage{hyperref}       % hyperlinks
\usepackage{url}            % simple URL typesetting
\usepackage{booktabs}       % professional-quality tables
\usepackage{amsmath,amsfonts,amssymb,amsthm}       % blackboard math symbols
\usepackage{nicefrac}       % compact symbols for 1/2, etc.
\usepackage{microtype}      % microtypography
\usepackage{multirow}
\usepackage{bbm}

\title{Graph Embedding with Personalized Context Distribution}


% The \author macro works with any number of authors. There are two commands
% used to separate the names and addresses of multiple authors: \And and \AND.
%
% Using \And between authors leaves it to LaTeX to determine where to break the
% lines. Using \AND forces a line break at that point. So, if LaTeX puts 3 of 4
% authors names on the first line, and the last on the second line, try using
% \AND instead of \And before the third author name.

\author{
Di Huang, Yuzhong Huang, Kexuan Sun, Zihao He}

\begin{document}



\maketitle

\begin{abstract}
    We propose a graph embedding method with node-personalized context distribution. 
\end{abstract}

\section{Introduction}
The efficacy of graph embedding, which entails vector representation for each node in the graph while preserving its structural information, has been demonstrated in various downstream applications such as node classification~\cite{} and link prediction~\cite{}.
Recently, dramatic advancements have been made in many existing works, such as DeepWalk~\cite{} and node2vec~\cite{}. 
% The usefulness of graph embeddings has been shown in various downstream applications, such as node classification and link prediction. Therefore, there have been dramatic advancements in learning graph embedding, such as DeepWalk~\cite{} and node2vec~\cite{}.

As a matter of fact, a vast majority of these methods that stem from random walk significantly depend on settings of their hyper-parameters, such as the length of walk $C$ that may vary among different graphs. As a result, the tuning of $C$ involves tremendous trial-and-error and is ususally time-consuming.
Fortunately, WatchYourStep~\cite{} extends a pre-defined $C$ to a trainable attention-based context distribution parametrized leveraging a learnable vector $\mathbf{Q}$, which aims to capture the subtle but non-trivial heterogeneity hidden in topology through an automatic manner. Such a graph attention model manages to reduce the error by 20\% to 40\%~\cite{}.

In spite of its capacity of considering the dissimilarity of different graphs, WatchYourStep fails to take the difference between nodes into consideration. Indeed, nodes from the same graph may have distinct characteristics and thus play distinct roles. For example, a small number of nodes can have much higher degree than any other nodes.

Instead of learning an attention-based context distribution for a graph, we therefore propose a graph embedding method that learns a context distribution for each individual node. We parameterize a context distribution into three different formats and compare them with other state-of-the-art graph embedding methods on both link prediction and node classification tasks. The experimental results indicate the superiority of our proposed method and may also potentially open up a new research direction in other domains such as natural language processing. 

\section{Methodology}


\subsection{Objective Function}
Similar with WatchYourStep, we use negative log graph likelihood(NLGL) as our log function:

$\displaystyle\min_{\mathbf{L},\mathbf{R},\mathbf{Q}}\alpha(\|\mathbf{L}\|_2^2 + \|\mathbf{R}\|_2^2 ) + \beta\|\mathbf{Q}\|_2^2 +\|-\mathbb{E}[\mathbf{D};\mathbf{Q}]\circ\log(\sigma(\mathbf{L}\times\mathbf{R}^\intercal)) -\mathbbm{1}[\mathbf{A}=0]\circ\log(1-\sigma(\mathbf{L}\times\mathbf{R}^\intercal)) \|_1$
where $\mathbb{E}[\mathbf{D};\mathbf{Q}] = \tilde{\mathbf{P}}^{(0)}\displaystyle\sum_{k=1}^{C}p(Q)_k(\mathcal{T})^k$. A is graph adjacent matrix and $\tilde{\mathbf{P}}^{(0)}$ is diagonal matrix notifying start points of random walks. $\mathbf{L}, \mathbf{R} \in \mathbb{R}^{|V|\times \frac{d}{2}}$ are left and right embedding respectively. They can be independently learnt or be equal. $\alpha \in \mathbb{R}$, $\beta \in \mathbb{R}$ are regularization coefficients for embedding and attention parameters. Instead of simulating random walks to calculate co-occurrence matrix $\mathbf{D}$ such as Node2Vec, we use transition matrix $\mathcal{T}$ to calculate $\mathbb{E}[\mathbf{D};\mathbf{Q}]$, the expectation of $\mathbf{D}$ between nodes on random walks. $\mathbf{Q}$ is defined as distribution parameters and $p(\mathbf{Q})$ is context distribution. 

\subsection{Attention Distribution}
We formulate distribution parameter $\mathbf{Q}$ and its mechanism to learn personalized context distribution $p(\mathbf{Q})$ in three ways as below.

\textbf{Freeform}.
$\mathbf{Q} \in \mathbb{R}^{|V|\times{C}}$ is a trainable parameter learned by the model. $p(\mathbf{Q}) = \text{softmax}(Q)$. There is no constraint on the learned attention distribution.

\textbf{Exponential distribution}.
Each row in $\mathbf{Q}$ is derived from an exponential distribution, parameterized by $\mathbf{q_i}$. For node $i$, $\mathbf{Q_i} = (\mathbf{q_i}^0, \mathbf{q}_i^1, \mathbf{q}_i^2, \cdots, \mathbf{q}_i^{C-1})$, $\mathbf{q}_i>0$. $p(\mathbf{Q_i}) = \frac{\mathbf{Q_i}}{\sum \mathbf{Q_i}}$. We use sum normalize instead of softmax to maintain $p(\mathbf{Q_i})$ as an exponential distribution. Depends on the value of $\mathbf{q}_i$, it could be monotonically increasing or decreasing.

\textbf{Quadratic distribution}.
Each row in $\mathbf{Q}$ is derived from a quadratic distribution, parameterized by $\mathbf{a_i, b_i, c_i}$. Denote $f_i(x) = a_ix^2+b_ix+c_i$. Then for node i, $\mathbf{Q_i} = (f_i(1), f_i(2), f_i(3), \cdots, f_i(C))$. $p(\mathbf{Q_i}) = \text{softmax}(\mathbf{Q_i})$. Depends on the value of $a_i, b_i, c_i$, it could be monotonically increasing, decreasing, or non-monotonic.

We also explore other forms of attention distribution, like gamma distribution, or more generally, a function of node embedding. Due to their mediocre experiment results and space limit, we do not include them in the discussion.

\section{Experiments And Analysis}
\subsection{Data Description}
As shown in Table \ref{tab:datasets}, we select 7 benchmark graphs for link prediction and node classification tasks. PPI is a network of biological interactions between proteins and labels are the functionality of proteins. Cora, Citeseer and Ca-Hepth are citation networks between scientific publications. Wikipedia is the cooccurrence network of words appearing in the first million bytes of Wikipedia dump while the labels are Part-of-Speech(POS) tags. Wikivote is a voting network between Wikipedia users. Soc-facebook contains social circles(friends list) on Facebook. 

\begin{table}[ht]
\caption{Statistics of benchmark datasets in the experiment}
\label{tab:datasets}
\centering
\begin{tabular}{c|c|c|c|c}
\toprule
Dataset & Nodes  & Edges & Is Directed & Classes\\
\midrule
PPI & 3,890  & 38,839 & Undirected & 50\\
Cora & 2,708  & 5,429 & Directed & 7\\
%BlogCatalog & 10,312 & 333,983 & Directed & 39\\
Citeseer & 3,312 & 4,660 & Undirected & 6\\
Wikipedia & 4,777 & 92,512 & Undirected & 40 \\
Ca-Hepth & 8,638 & 24,827 & Undirected & -\\
Wikivote & 7,066 & 103,663 & Directed & -\\
Soc-facebook & 4,039 & 88,234 & Undirected & -\\
\bottomrule
\end{tabular}
\end{table}

\subsection{Baseline Graph Embedding Methods}


\textbf{HOPE}: It learns node embedding by minimizing $||S - Y_sY_t^\intercal||_F^2$, where $S$ is similarity matrix between nodes. $S$ can be measured using different similarity metrics such as Katz Index, Rooted Page Rank. 

\textbf{Node2Vec}: It is a random walk based  approach with two parameters controlling walk directions. It preserves the high-order proximity between nodes by maximizing the probability of node co-occurrence on fixed-length walks. 

\textbf{WatchYourStep}: Built upon random walk method, it replaces context hyper-parameters such as walk length and the number of walks with a shared trainable attention vector to learn context distributions.




\subsection{Experiment Setup And Evaluation Metrics}
We evaluate the generated embedding with personalized attention on two tasks -- link prediction and node classification. We keep embedding dimension as 128 for all embedding methods. In random walks based methods node2vec,  the window size in random walks of 10. We use 80 random walks starting from each node. For link prediction, we hide 50\% links on the original graph for test and embed the graph with the rest 50\% edges as training. In node classification, we generate node embedding from the original graph without edge splitting. Then we use embedding of 70\% training nodes as feature vectors and apply logistic regression as classifier to predict labels for 30\% test nodes. We use MAP to evaluate link prediction and Macro-F1/Micro-F1 to evaluate node classification respectively. Since splitting training and testing sets utilizes a random number generator, we run 5 times varying its seed and report the average scores from all runs.


% We use Adam \ref{} optimizer with a learning rate of 0.2. We choose the weight decay to be 1e-6. The regularization number $\alpha$ for the adjacency matrix is 0.5. Unless otherwise specified. During training, the maximum iteration is 1,000 and we stop the training if the best training loss stagnates for more than 100 iterations. 
 
 %% MAP is correct. Please make sure all letters are capital. I haven't seen mAP.
\textbf{MAP}(Mean Average Precision). It computes the average prediction precision across all nodes. $MAP=\frac{\sum_i AP(i)}{|V|}$,
where $AP(i) = \frac{\sum_{k=1}^K P@k(i) \cdot \mathbb{I}\{E_{pred_i}(k) \in E_{obs_i}\}}{|\{k: E_{pred_i}(k) \in E_{obs_i}\}|}$, $P@k(i) = \frac{|E_{pred_i}(1:k) \cap E_{obs_i}|}{k}$, $E_{pred_i}$ and $E_{obs_i}$ are the predicted and observed edges for node $i$ respectively. We set $K=100$ in our experiments.

\textbf{Micro-F1/Macro-F1}. In multi-label classification, Macro-F1 is the average F1 score of all classes while Micro-F1 is defined as global F1 score by calculating the total true positives, false negatives and false positives with equal weight on each class.



 
\subsection{Link Prediction}
In this task, independent left and right embedding are used for each node.
Table \ref{tab:results_lp} shows the results of link prediction for all compared methods with regard to mAP.
We found personalized freeform attention method is better than WatchYourStep, but is still not optimal. One reason is it does not have any constraint on the learned attention distribution, so it is prone to over-fit, and for some less visited nodes, it don't have enough data to learn a good distribution. Instead, our proposed personalized exponential attention is superior on most datasets.

\begin{table}
\caption{MAP of link prediction using different embedding methods}
\label{tab:results_lp}
\centering
\begin{tabular}{cccc|ccc}
\toprule
\multirow{2}{*}{\textbf{Dataset}} & \multicolumn{3}{c}{\textbf{Baselines}} & \multicolumn{3}{c}{\textbf{Personalized}} \\
 & \textbf{HOPE} & \textbf{N2V} & \textbf{WYS} & \textbf{Freeform} & \textbf{Exponential} & \textbf{Quadratic}\\
\midrule
PPI & 12.01  & 1.91 & 19.28 & 19.39 & \textbf{19.41} & 17.94\\
Cora & 7.30  & 2.54 & 10.22 & 10.33 & \textbf{10.75} & 9.95\\
Citeseer & 14.38  & 6.46  & 29.49 & \textbf{29.63} & 29.58 & 29.28 \\
Wikipedia & 3.47  & 1.53  & 7.10 & 6.81 & \textbf{7.68} & 6.82\\
Ca-Hepth & 13.63  & 14.01  & 39.67 & 37.55 & \textbf{43.00} & 38.26\\
Wikivote & 6.92  & 6.70  & 12.81 & 12.75 & \textbf{13.46} & 11.19\\
Soc-facebook & 41.48  & 22.47 & 53.77 & 51.82 & \textbf{54.37} & 52.62\\

\bottomrule
\end{tabular}
\end{table}



\subsection{Node Classification}
In this task, a shared left and right embedding are used for directed graph, independent left and right embedding are used for undirected graph.
Table \ref{tab:results_nc} shows the results of node classification for all compared methods with regard to Micro-F1/Macro-F1.
We found the optimal form of attention is different on different tasks and datasets. The proposed personalized quadratic attention generally works better, since it allow non-monotonic distribution of the attention weights.

\begin{table}
\caption{Micro-F1/Macro-F1 of node classification using different embedding methods}
\label{tab:results_nc}
\centering
\begin{tabular}{cccc|ccc}
\toprule
\multirow{2}{*}{\textbf{Dataset}} & \multicolumn{3}{c}{\textbf{Baselines}} & \multicolumn{3}{c}{\textbf{Personalized}} \\
 & \textbf{HOPE}  & \textbf{N2V} & \textbf{WYS} & \textbf{Freeform} & \textbf{Exponential} & \textbf{Quadratic}\\
\midrule
PPI & 8.46/3.05 & 17.08/13.47  & 20.50/17.16 & 20.25/17.12 & \textbf{20.71/17.63} & 20.48/17.30\\
Cora & 28.82/6.39 & 44.37/35.76  & 76.97/74.08 & 77.00/74.28 & 76.85/74.28 & \textbf{77.93/75.15}\\
Citeseer & 24.58/6.65  & 55.86/46.01 & 74.20/66.99 & \textbf{74.43/67.66} & 74.32/67.21 & 74.26/67.03\\
Wikipedia & 41.98/3.90  & 41.21/4.44 & 43.81/8.26 & 44.03/8.41 & 44.86/8.36 & \textbf{49.03/10.70}\\

\bottomrule
\end{tabular}
\end{table}


\subsection{Personalized Context Analysis}

From Table \ref{tab:results_lp} and Table \ref{tab:results_nc}, we can see that graph embedding methods with personalized context distribution significantly outperform the global method. We argue that the superiority of our method stems from the dissimilarity of nodes in a graph. For example, some nodes with a higher degree may have a larger context to look at while other more disconnected nodes may have a smaller one. As a matter of fact, a global context distribution presumes the homogeneity over various nodes, which tend to not hold water. In contrast, our method endows each node with sufficient autonomy to explore their personalized optimal context distribution parametrized by only a few learnable parameters, the quantity of which is fairly tractable.

To delve deeper into what vantage our personalized embedding methods provide for each node, we visualize the statistics of the parameters learned by personalized exponential distribution and personalized quadratic distribution, as shown in Figure \ref{} and Table \ref{}.
% Table for each combination of parameters, table for statistics (mean, average, and etc) of those parameters.

\textbf{Analysis of personalized exponential distribution.} 

\begin{table}
\caption{Statistics of q}
\label{tab:analysis_q}
\centering
\begin{tabular}{ccccccc}
\toprule
Dataset & Min  & Max & Mean & Median & Skewness & Kurtosis \\
\midrule
PPI & 12.01  & 1.91 & 19.28 & 19.39 \\
Cora & 7.30  & 2.54 & 10.22 & 10.33 \\
Citeseer & 14.38  & 6.46  & 29.49 & 123 \\
Wikipedia & 3.47  & 1.53  & 7.10 & 6.81 \\
Ca-Hepth & 13.63  & 14.01  & 39.67 & 37.55 \\
Wikivote & 6.92  & 6.70  & 12.81 & 12.75\\
Soc-facebook & 41.48  & 22.47 & 53.77 & 51.82 \\

\bottomrule
\end{tabular}
\end{table}

\textbf{Analysis of personalized quadratic distribution.} blablabla


\section{Conclusions and Future Work}
In this paper, we proposed a graph embedding method which extends WatchYourStep by personalizing context distribution for each individual node. We demonstrate the superiority of our method by evaluating the learned embedding on two downstream tasks: link prediction and multi-class node classification. In both tasks, our method outperforms all other compared methods on all benchmark datasets. 

One possible future work is to initialize embedding of each node based on its role in the graph. Another direction of future work is to apply our method in other domains such as natural language processing. For example, word embedding can be potentially learned in a personalized way.

% Please read the instructions below carefully and follow them faithfully.

% \subsection{Style}

% Papers to be submitted to NeurIPS 2019 must be prepared according to the
% instructions presented here. Papers may only be up to eight pages long,
% including figures. Additional pages \emph{containing only acknowledgments and/or
%   cited references} are allowed. Papers that exceed eight pages of content
% (ignoring references) will not be reviewed, or in any other way considered for
% presentation at the conference.

% The margins in 2019 are the same as since 2007, which allow for $\sim$$15\%$
% more words in the paper compared to earlier years.

% Authors are required to use the NeurIPS \LaTeX{} style files obtainable at the
% NeurIPS website as indicated below. Please make sure you use the current files
% and not previous versions. Tweaking the style files may be grounds for
% rejection.

% \subsection{Retrieval of style files}

% The style files for NeurIPS and other conference information are available on
% the World Wide Web at
% \begin{center}
%   \url{http://www.neurips.cc/}
% \end{center}
% The file \verb+neurips_2019.pdf+ contains these instructions and illustrates the
% various formatting requirements your NeurIPS paper must satisfy.

% The only supported style file for NeurIPS 2019 is \verb+neurips_2019.sty+,
% rewritten for \LaTeXe{}.  \textbf{Previous style files for \LaTeX{} 2.09,
%   Microsoft Word, and RTF are no longer supported!}

% The \LaTeX{} style file contains three optional arguments: \verb+final+, which
% creates a camera-ready copy, \verb+preprint+, which creates a preprint for
% submission to, e.g., arXiv, and \verb+nonatbib+, which will not load the
% \verb+natbib+ package for you in case of package clash.

% \paragraph{Preprint option}
% If you wish to post a preprint of your work online, e.g., on arXiv, using the
% NeurIPS style, please use the \verb+preprint+ option. This will create a
% nonanonymized version of your work with the text ``Preprint. Work in progress.''
% in the footer. This version may be distributed as you see fit. Please \textbf{do
%   not} use the \verb+final+ option, which should \textbf{only} be used for
% papers accepted to NeurIPS.

% At submission time, please omit the \verb+final+ and \verb+preprint+
% options. This will anonymize your submission and add line numbers to aid
% review. Please do \emph{not} refer to these line numbers in your paper as they
% will be removed during generation of camera-ready copies.

% The file \verb+neurips_2019.tex+ may be used as a ``shell'' for writing your
% paper. All you have to do is replace the author, title, abstract, and text of
% the paper with your own.

% The formatting instructions contained in these style files are summarized in
% Sections \ref{gen_inst}, \ref{headings}, and \ref{others} below.

% \section{General formatting instructions}
% \label{gen_inst}

% The text must be confined within a rectangle 5.5~inches (33~picas) wide and
% 9~inches (54~picas) long. The left margin is 1.5~inch (9~picas).  Use 10~point
% type with a vertical spacing (leading) of 11~points.  Times New Roman is the
% preferred typeface throughout, and will be selected for you by default.
% Paragraphs are separated by \nicefrac{1}{2}~line space (5.5 points), with no
% indentation.

% The paper title should be 17~point, initial caps/lower case, bold, centered
% between two horizontal rules. The top rule should be 4~points thick and the
% bottom rule should be 1~point thick. Allow \nicefrac{1}{4}~inch space above and
% below the title to rules. All pages should start at 1~inch (6~picas) from the
% top of the page.

% For the final version, authors' names are set in boldface, and each name is
% centered above the corresponding address. The lead author's name is to be listed
% first (left-most), and the co-authors' names (if different address) are set to
% follow. If there is only one co-author, list both author and co-author side by
% side.

% Please pay special attention to the instructions in Section \ref{others}
% regarding figures, tables, acknowledgments, and references.

% \section{Headings: first level}
% \label{headings}

% All headings should be lower case (except for first word and proper nouns),
% flush left, and bold.

% First-level headings should be in 12-point type.

% \subsection{Headings: second level}

% Second-level headings should be in 10-point type.

% \subsubsection{Headings: third level}

% Third-level headings should be in 10-point type.

% \paragraph{Paragraphs}

% There is also a \verb+\paragraph+ command available, which sets the heading in
% bold, flush left, and inline with the text, with the heading followed by 1\,em
% of space.

% \section{Citations, figures, tables, references}
% \label{others}

% These instructions apply to everyone.

% \subsection{Citations within the text}

% The \verb+natbib+ package will be loaded for you by default.  Citations may be
% author/year or numeric, as long as you maintain internal consistency.  As to the
% format of the references themselves, any style is acceptable as long as it is
% used consistently.

% The documentation for \verb+natbib+ may be found at
% \begin{center}
%   \url{http://mirrors.ctan.org/macros/latex/contrib/natbib/natnotes.pdf}
% \end{center}
% Of note is the command \verb+\citet+, which produces citations appropriate for
% use in inline text.  For example,
% \begin{verbatim}
%   \citet{hasselmo} investigated\dots
% \end{verbatim}
% produces
% \begin{quote}
%   Hasselmo, et al.\ (1995) investigated\dots
% \end{quote}

% If you wish to load the \verb+natbib+ package with options, you may add the
% following before loading the \verb+neurips_2019+ package:
% \begin{verbatim}
%   \PassOptionsToPackage{options}{natbib}
% \end{verbatim}

% If \verb+natbib+ clashes with another package you load, you can add the optional
% argument \verb+nonatbib+ when loading the style file:
% \begin{verbatim}
%   \usepackage[nonatbib]{neurips_2019}
% \end{verbatim}

% As submission is double blind, refer to your own published work in the third
% person. That is, use ``In the previous work of Jones et al.\ [4],'' not ``In our
% previous work [4].'' If you cite your other papers that are not widely available
% (e.g., a journal paper under review), use anonymous author names in the
% citation, e.g., an author of the form ``A.\ Anonymous.''

% \subsection{Footnotes}

% Footnotes should be used sparingly.  If you do require a footnote, indicate
% footnotes with a number\footnote{Sample of the first footnote.} in the
% text. Place the footnotes at the bottom of the page on which they appear.
% Precede the footnote with a horizontal rule of 2~inches (12~picas).

% Note that footnotes are properly typeset \emph{after} punctuation
% marks.\footnote{As in this example.}

% \subsection{Figures}

% \begin{figure}
%   \centering
%   \fbox{\rule[-.5cm]{0cm}{4cm} \rule[-.5cm]{4cm}{0cm}}
%   \caption{Sample figure caption.}
% \end{figure}

% All artwork must be neat, clean, and legible. Lines should be dark enough for
% purposes of reproduction. The figure number and caption always appear after the
% figure. Place one line space before the figure caption and one line space after
% the figure. The figure caption should be lower case (except for first word and
% proper nouns); figures are numbered consecutively.

% You may use color figures.  However, it is best for the figure captions and the
% paper body to be legible if the paper is printed in either black/white or in
% color.

% \subsection{Tables}

% All tables must be centered, neat, clean and legible.  The table number and
% title always appear before the table.  See Table~\ref{sample-table}.

% Place one line space before the table title, one line space after the
% table title, and one line space after the table. The table title must
% be lower case (except for first word and proper nouns); tables are
% numbered consecutively.

% Note that publication-quality tables \emph{do not contain vertical rules.} We
% strongly suggest the use of the \verb+booktabs+ package, which allows for
% typesetting high-quality, professional tables:
% \begin{center}
%   \url{https://www.ctan.org/pkg/booktabs}
% \end{center}
% This package was used to typeset Table~\ref{sample-table}.

% \begin{table}
%   \caption{Sample table title}
%   \label{sample-table}
%   \centering
%   \begin{tabular}{lll}
%     \toprule
%     \multicolumn{2}{c}{Part}                   \\
%     \cmidrule(r){1-2}
%     Name     & Description     & Size ($\mu$m) \\
%     \midrule
%     Dendrite & Input terminal  & $\sim$100     \\
%     Axon     & Output terminal & $\sim$10      \\
%     Soma     & Cell body       & up to $10^6$  \\
%     \bottomrule
%   \end{tabular}
% \end{table}

% \section{Final instructions}

% Do not change any aspects of the formatting parameters in the style files.  In
% particular, do not modify the width or length of the rectangle the text should
% fit into, and do not change font sizes (except perhaps in the
% \textbf{References} section; see below). Please note that pages should be
% numbered.

% \section{Preparing PDF files}

% Please prepare submission files with paper size ``US Letter,'' and not, for
% example, ``A4.''

% Fonts were the main cause of problems in the past years. Your PDF file must only
% contain Type 1 or Embedded TrueType fonts. Here are a few instructions to
% achieve this.

% \begin{itemize}

% \item You should directly generate PDF files using \verb+pdflatex+.

% \item You can check which fonts a PDF files uses.  In Acrobat Reader, select the
%   menu Files$>$Document Properties$>$Fonts and select Show All Fonts. You can
%   also use the program \verb+pdffonts+ which comes with \verb+xpdf+ and is
%   available out-of-the-box on most Linux machines.

% \item The IEEE has recommendations for generating PDF files whose fonts are also
%   acceptable for NeurIPS. Please see
%   \url{http://www.emfield.org/icuwb2010/downloads/IEEE-PDF-SpecV32.pdf}

% \item \verb+xfig+ "patterned" shapes are implemented with bitmap fonts.  Use
%   "solid" shapes instead.

% \item The \verb+\bbold+ package almost always uses bitmap fonts.  You should use
%   the equivalent AMS Fonts:
% \begin{verbatim}
%   \usepackage{amsfonts}
% \end{verbatim}
% followed by, e.g., \verb+\mathbb{R}+, \verb+\mathbb{N}+, or \verb+\mathbb{C}+
% for $\mathbb{R}$, $\mathbb{N}$ or $\mathbb{C}$.  You can also use the following
% workaround for reals, natural and complex:
% \begin{verbatim}
%   \newcommand{\RR}{I\!\!R} %real numbers
%   \newcommand{\Nat}{I\!\!N} %natural numbers
%   \newcommand{\CC}{I\!\!\!\!C} %complex numbers
% \end{verbatim}
% Note that \verb+amsfonts+ is automatically loaded by the \verb+amssymb+ package.

% \end{itemize}

% If your file contains type 3 fonts or non embedded TrueType fonts, we will ask
% you to fix it.

% \subsection{Margins in \LaTeX{}}

% Most of the margin problems come from figures positioned by hand using
% \verb+\special+ or other commands. We suggest using the command
% \verb+\includegraphics+ from the \verb+graphicx+ package. Always specify the
% figure width as a multiple of the line width as in the example below:
% \begin{verbatim}
%   \usepackage[pdftex]{graphicx} ...
%   \includegraphics[width=0.8\linewidth]{myfile.pdf}
% \end{verbatim}
% See Section 4.4 in the graphics bundle documentation
% (\url{http://mirrors.ctan.org/macros/latex/required/graphics/grfguide.pdf})

% A number of width problems arise when \LaTeX{} cannot properly hyphenate a
% line. Please give LaTeX hyphenation hints using the \verb+\-+ command when
% necessary.

% \subsubsection*{Acknowledgments}

% Use unnumbered third level headings for the acknowledgments. All acknowledgments
% go at the end of the paper. Do not include acknowledgments in the anonymized
% submission, only in the final paper.

\section*{References}

% References follow the acknowledgments. Use unnumbered first-level heading for
% the references. Any choice of citation style is acceptable as long as you are
% consistent. It is permissible to reduce the font size to \verb+small+ (9 point)
% when listing the references. {\bf Remember that you can use more than eight
%   pages as long as the additional pages contain \emph{only} cited references.}
% \medskip

\small

[1] Alexander, J.A.\ \& Mozer, M.C.\ (1995) Template-based algorithms for
connectionist rule extraction. In G.\ Tesauro, D.S.\ Touretzky and T.K.\ Leen
(eds.), {\it Advances in Neural Information Processing Systems 7},
pp.\ 609--616. Cambridge, MA: MIT Press.

\end{document}
